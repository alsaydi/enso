
\section{\Enso: Self-describing language specification}

\comments{
- What is Enso
  - Enso is an external DSL workbench (toolkit for creating ext DSLs...)
  - Each DSL is defined by a metamodel (i dun like this word, use abstract syntax), aka schema written by the programmer
    - metamodel is amanaged data --> zzz...
      - provide services such as inverses, etc (do i care? this is a program, not a data model, not even program state!)
  - optionally can be edited textually or graphically depending on whether a grammar or a \emph{stencil}, a graphical blahblah specification language
    - decoupling gives:
      - representation freedom
      - 
- internally
  - 'everything is a language'
  - each component, schema, grammar, stencil, diff, etc, is a DSL.
  - bootstrapped from a minimal schema and grammar model
  - metametamodels are explicitly represented and editable
  - advantages?
    - the same tools for building DSLs as using DSLs
    - "multi-level metamodeling" ?? able to ????
  - 
}

\Enso is an external DSL workbench with both textual and graphical editing capabilities. Each DSL is defined by its \emph{schema}, a model of its abstract syntax which conforms to a meta-metamodel loosely similar to MOF. The choice of textual and graphical representation allows

Additionally, the meta-metamodel in \Enso is editable and loaded at run-time. This allows the workbench to be extensible.

\subsection{'Everything is a language'}


At its core, each language created in \Enso is defined by a model of its abstract syntax, known as its \emph{schema}. This schema can be created and edited either graphically or textually.



