\comments{
Contributions of this work:
- Composition
  - adding new data types and operations
- Cross-cutting modifications
  - operation inheritance
    - compound operation composition (of which inheritance is special case)
- Meta-level
  - Schema and grammar as well 
- Feature-oriented 'packaging' -> gotta give this a better name

How we do it:
- s

}

\section{Proposed solution}

Our approach enforce the rigor of a formal specification on hand-written code. By designing a 

\begin{enumerate}
\item x
\item y
\end{enumerate}

\subsection{\Enso: Self-describing language specification}

\comments{
- What is Enso
  - Enso is an external DSL workbench (toolkit for creating ext DSLs...)
  - Each DSL is defined by a metamodel (i dun like this word, use abstract syntax), aka schema written by the programmer
    - metamodel is amanaged data --> zzz...
      - provide services such as inverses, etc (do i care? this is a program, not a data model, not even program state!)
  - optionally can be edited textually or graphically depending on whether a grammar or a \emph{stencil}, a graphical blahblah specification language
- internally
  - each component, schema, grammar, stencil, diff, etc, is a DSL.
  - bootstrapped from a minimal schema and grammar model
  - advantages?
    - the same tools for building DSLs as using DSLs
}

At its core, each language created in \Enso is defined by a model of its abstract syntax, known as its \emph{schema}. This schema can be created and edited either graphically or textually.

\comments{
\subsection{Adding new data types and operations}
(i'm not going to let this guy get his own subsection
- interpreters defined as visitor mixins
  - simple dispatch system based on data type
    - has subtype polymorphism and a 'root' type <-- i dont know what to call this
- well-known concept that allows extension of ...

- internal visitor enforce composability
  - internal visitors are produced via composition
}

\subsection{Cross-cutting extensions}

\comments{
  
}

\subsubsection{Higher-order interpreter composition}

\subsection{Meta-level language extensions}

\subsection{Cross-model language features}

\comments{
- features
  - define "features"
- package different types of artefacts together
- because of meta-level this can be used both for dsl dev as well as dsl
}

\subsection{Tooling for extensible DSLs}





