% This is "sig-alternate.tex" V1.9 April 2009
% This file should be compiled with V2.4 of "sig-alternate.cls" April 2009
%
% This example file demonstrates the use of the 'sig-alternate.cls'
% V2.4 LaTeX2e document class file. It is for those submitting
% articles to ACM Conference Proceedings WHO DO NOT WISH TO
% STRICTLY ADHERE TO THE SIGS (PUBS-BOARD-ENDORSED) STYLE.
% The 'sig-alternate.cls' file will produce a similar-looking,
% albeit, 'tighter' paper resulting in, invariably, fewer pages.
%
% ----------------------------------------------------------------------------------------------------------------
% This .tex file (and associated .cls V2.4) produces:
%       1) The Permission Statement
%       2) The Conference (location) Info information
%       3) The Copyright Line with ACM data
%       4) NO page numbers
%
% as against the acm_proc_article-sp.cls file which
% DOES NOT produce 1) thru' 3) above.
%
% Using 'sig-alternate.cls' you have control, however, from within
% the source .tex file, over both the CopyrightYear
% (defaulted to 200X) and the ACM Copyright Data
% (defaulted to X-XXXXX-XX-X/XX/XX).
% e.g.
% \CopyrightYear{2007} will cause 2007 to appear in the copyright line.
% \crdata{0-12345-67-8/90/12} will cause 0-12345-67-8/90/12 to appear in the copyright line.
%
% ---------------------------------------------------------------------------------------------------------------
% This .tex source is an example which *does* use
% the .bib file (from which the .bbl file % is produced).
% REMEMBER HOWEVER: After having produced the .bbl file,
% and prior to final submission, you *NEED* to 'insert'
% your .bbl file into your source .tex file so as to provide
% ONE 'self-contained' source file.
%
% ================= IF YOU HAVE QUESTIONS =======================
% Questions regarding the SIGS styles, SIGS policies and
% procedures, Conferences etc. should be sent to
% Adrienne Griscti (griscti@acm.org)
%
% Technical questions _only_ to
% Gerald Murray (murray@hq.acm.org)
% ===============================================================
%
% For tracking purposes - this is V1.9 - April 2009

\documentclass{sig-alternate}

\begin{document}

\newcommand{\comments}[1]{}
\newcommand{\Enso}{Ens\={o}}

%
% --- Author Metadata here ---
\conferenceinfo{WOODSTOCK}{'97 El Paso, Texas USA}
%\CopyrightYear{2007} % Allows default copyright year (20XX) to be over-ridden - IF NEED BE.
%\crdata{0-12345-67-8/90/01}  % Allows default copyright data (0-89791-88-6/97/05) to be over-ridden - IF NEED BE.
% --- End of Author Metadata ---

\title{\Enso: Extensible interpreters for DSL workbenches}
%
% You need the command \numberofauthors to handle the 'placement
% and alignment' of the authors beneath the title.
%
% For aesthetic reasons, we recommend 'three authors at a time'
% i.e. three 'name/affiliation blocks' be placed beneath the title.
%
% NOTE: You are NOT restricted in how many 'rows' of
% "name/affiliations" may appear. We just ask that you restrict
% the number of 'columns' to three.
%
% Because of the available 'opening page real-estate'
% we ask you to refrain from putting more than six authors
% (two rows with three columns) beneath the article title.
% More than six makes the first-page appear very cluttered indeed.
%
% Use the \alignauthor commands to handle the names
% and affiliations for an 'aesthetic maximum' of six authors.
% Add names, affiliations, addresses for
% the seventh etc. author(s) as the argument for the
% \additionalauthors command.
% These 'additional authors' will be output/set for you
% without further effort on your part as the last section in
% the body of your article BEFORE References or any Appendices.

\numberofauthors{3} %  in this sample file, there are a *total*
% of EIGHT authors. SIX appear on the 'first-page' (for formatting
% reasons) and the remaining two appear in the \additionalauthors section.
%
\author{
% You can go ahead and credit any number of authors here,
% e.g. one 'row of three' or two rows (consisting of one row of three
% and a second row of one, two or three).
%
% The command \alignauthor (no curly braces needed) should
% precede each author name, affiliation/snail-mail address and
% e-mail address. Additionally, tag each line of
% affiliation/address with \affaddr, and tag the
% e-mail address with \email.
%
% 1st. author
\alignauthor
Ben Trovato\\
       \affaddr{Institute for Clarity in Documentation}\\
       \affaddr{1932 Wallamaloo Lane}\\
       \affaddr{Wallamaloo, New Zealand}\\
       \email{trovato@corporation.com}
% 2nd. author
\alignauthor
G.K.M. Tobin\\
       \affaddr{Institute for Clarity in Documentation}\\
       \affaddr{P.O. Box 1212}\\
       \affaddr{Dublin, Ohio 43017-6221}\\
       \email{webmaster@marysville-ohio.com}
% 3rd. author
\alignauthor Lars Th{\o}rv{\"a}ld\\
       \affaddr{The Th{\o}rv{\"a}ld Group}\\
       \affaddr{1 Th{\o}rv{\"a}ld Circle}\\
       \affaddr{Hekla, Iceland}\\
       \email{larst@affiliation.org}
}
% There's nothing stopping you putting the seventh, eighth, etc.
% author on the opening page (as the 'third row') but we ask,
% for aesthetic reasons that you place these 'additional authors'
% in the \additional authors block, viz.
% Just remember to make sure that the TOTAL number of authors
% is the number that will appear on the first page PLUS the
% number that will appear in the \additionalauthors section.

\maketitle
\begin{abstract}
\comments{
- What is this paper about (one-liner)
- What is the program
- What is the proposed solution
- What are the potential/perceived benefits
- What are the results/observations
}

This paper describes Enso, an domain-specific language (DSL) workbench with extensible semantics. Our goal is to defray the high costs of developing and managing the multiple DSLs associated with the language-oriented programming paradigm by promoting reusability



\end{abstract}

% A category with the (minimum) three required fields
\category{H.4}{Information Systems Applications}{Miscellaneous}
%A category including the fourth, optional field follows...
\category{D.2.8}{Software Engineering}{Metrics}[complexity measures, performance measures]

\terms{}

\keywords{Domain-specific languages, DSL workbench, Model-driven engineering}

\comments{
Problem: Engineering DSLs is hard
- requires a different degree of reuse than normal code
- challenges include:
  - unanticipated change [van Deursen98] -> spin this as 
    "we want to support dsl evolution" rather than the 
    whole co-evolution businness
  - many-dsl interaction, etc etc [France01]
  - reuse not good from empirical study [hermans09], 
    but it's model reuse, not language/metamodel reuse
- default solns (ie writing code) not good, because:
  - ?

problems:
- initial cost of developing dsls
  - time and manpower
  - shift in required skill set [vanDeursen98]
- sustained cost of maintenance
  - system requirements changes
  - role of dsl expand into new products, new features
- dsl interaction
  - enforce referential integrity between related concepts


so far:
- relatively successful at integrating models
  - emf, model weaving
  - extensible parsing techniques
- semantics, not so successful
  - ways of defining operational semantics for 
  - code generators
  - 
}

\section{Introduction}

Domain specific languages (DSLs) are languages specialized for a particular problem domain. A set of specialists, known as /toolsmiths/, first construct DSLs encapsulating the relevant domain knowledge, which application programmers can then use to build specific applications. Examples of DSLs include SQL for database management, Yacc for text parsing, XUL and HTML for user interfaces, Promela and Alloy for program verification, and Verilog for hardware description, among many others. DSLs allow application programmers to work with a vocabulary and language structure similar to that of the domain expert, thus reducing the syntactic misalignment between problem definition and solution specification. Language-oriented programming [Ward94] is a software engieering paradigm which systematically deploys DSLs as the primary artefact of program organization, allowing domain knowledge to be remain encapsulated across traditional module boundaries of of classes and functions. Model driven engineering [OMG01] is a variation on the same concept that substitutes concrete textual DSLs with /models/ of their abstract represention.

Unfortunately these advantages do not come for free. The initial development of the DSL incur an additional one-time cost over directly implementing the application. Apart from time and manpower, DSL development also necessitate a shift in programmer skill sets [vanDeursen98], not just in learning the relevant DSL development tools but also in mastering the patterns and prinicples of good language design. Moreover, even though one of the goal of DSLs as an "enabler of reuse" [Mernik05], the added range of artefacts from grammars to interpreters to tools, and their often monolithic nature, <this point needs to be substantiated> ironically hinders reuse in the development of the DSLs themselves. < .. somewhere here mention how code generators exacerbate the problem by introducing another level of indirection, hard-to-detect interference, etc>

The other major cost comes from maintaining these DSLs. Like other software systems, DSLs are seldom static and evolve according to requirement changes. Even if the domain logic remains constant, the DSL's responsibility might increase over time as it becomes used more widely [?]. The robustness challenge for DSL developers here is to tolerate slight variations across versions, including building checkers to determine if the particular variation is allowed. In the ideal case, a DSL may also refined modularly, such that any combination of variations can be selected without an exponential increase in code.

Maintainence is particularly difficult in environments involving multiple DSLs which, under the language-oriented paradigm, is now the norm. Changes in one language can 





\section{Motivating Example}

\comments{
Motivating example: Web development framework

EnsoWeb is good example because:
- multiple DSL environment
  - heavy reuse, eg expressions
    - avoid code duplication and replicated maintainence
    - integration concerns, ie my expression is the same as yours
  - modular feature composition (this is a *very* dangerous buzzword combination!)
    - eg secure, db, secure+db, etc
  - crosscutting features, eg security
  - tooling (we don't really have this yet)
- 'real' application --- (how many klocs? <-- is this even a good gauge?)

Goals:
- Build a set of interpreters that:
  - allow library-like reuse of sub-languages
    - so that language components can be made comparable across DSLs
  - accommodate pervasive interactions between DSLs
  - robust under evolution
    - you can change, add or remove individual pieces

Current:
- Model level composition
  - 
}

EnsoWeb is a web development framework created with Enso loosely following the Model-View-Presenter architecture. It comprises a number of DSLs for defining data models, web interfaces, security policies, and database schemas. These DSLs share common language components, like expressions, which should be reused. Reusing language components also makes it easier to pass these share pieces between DSLs. Some DSLs have dependencies; in EnsoWeb all DSLs depend on the data model. Others have semantic operations that interact with each other in crosscutting and possibly intricate ways. For instance, security policies limit the data the web interface can access, and the web interface determines database prefetching strategies. There is also the issue of modularity: we want to be able to enable any combination of optional features like security and database pre-fetching with a linear amount of source code changes. Furthermore, if EnsoWeb is to be used in a serious development environment, tool support such as debuggers and versioning specialized for each DSL are necessary and should be produced with minimal coding. These concerns in EnsoWeb are fairly typical of a DSL-based framework with multiple languages and commercial systems, like Microsoft's Oslo system, may boast even more DSLs.

\comments{Some diagram to illustrate information flow between components and the DSLs that describe them}

Our goal is to build a set of interpreters for these DSLs that accomodate close interaction between DSLs, yet remains robust under evolution and allows individual components to be modified modularly.

* Enable library-like reuse of sub-languages, such as the expression language. 

* Accomodate pervasive interactions between DSLs that remain robust under evolution

* Modular composition of optional features like security and database pre-fetching

* Tooling




\input{related}

<%
Contributions of this work:
- 

How we do it:
- s

%>




\comments{This section better merged into Solution}


\input{conclusion}


%
% The following two commands are all you need in the
% initial runs of your .tex file to
% produce the bibliography for the citations in your paper.
\bibliographystyle{abbrv}
\bibliography{interpreter}  % sigproc.bib is the name of the Bibliography in this case
% You must have a proper ".bib" file
%  and remember to run:
% latex bibtex latex latex
% to resolve all references
%
% ACM needs 'a single self-contained file'!
%

\end{document}

