\section{Motivating Example}

\comments{
Motivating example: Web development framework

EnsoWeb is good example because:
- multiple DSL environment
  - heavy reuse, eg expressions
    - avoid code duplication and replicated maintainence
    - integration concerns, ie my expression is the same as yours
  - modular feature composition (this is a *very* dangerous buzzword combination!)
    - eg secure, db, secure+db, etc
  - crosscutting features, eg security
  - tooling (we don't really have this yet)
- 'real' application --- (how many klocs? <-- is this even a good gauge?)

Goals:
- Build a set of interpreters that:
  - allow library-like reuse of sub-languages
    - so that language components can be made comparable across DSLs
  - accommodate pervasive interactions between DSLs
  - robust under evolution
    - you can change, add or remove individual pieces

Current:
- Model level composition
  - 
}

EnsoWeb is a web development framework created with Enso loosely following the Model-View-Presenter architecture. It comprises a number of DSLs for defining data models, web interfaces, security policies, and database schemas. These DSLs share common language components, like expressions, which should be reused. Reusing language components also makes it easier to pass these share pieces between DSLs. Some DSLs have dependencies; in EnsoWeb all DSLs depend on the data model. Others have semantic operations that interact with each other in crosscutting and possibly intricate ways. For instance, security policies limit the data the web interface can access, and the web interface determines database prefetching strategies. There is also the issue of modularity: we want to be able to select any combination of optional features like security and database pre-fetching with a linear amount of source code changes. Furthermore, if EnsoWeb is to be used in a serious development environment, tool support such as debuggers and versioning specialized for each DSL are necessary and should be produced with minimal coding. These concerns in EnsoWeb are fairly typical of a DSL-based framework with multiple languages and commercial systems, like Microsoft's Oslo system, may boast even more DSLs.

\comments{Some diagram to illustrate information flow between components and the DSLs that describe them}

Our goal is to build a set of interpreters for these DSLs that accomodate close interaction between DSLs, yet remains robust under evolution and allows individual components to be modified modularly.

* Enable library-like reuse of sub-languages, such as the expression language. 

* Accomodate pervasive interactions between DSLs that remain robust under evolution

* Modular composition of optional features like security and database pre-fetching

* Tooling


