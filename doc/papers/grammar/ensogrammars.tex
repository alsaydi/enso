\documentclass[a4paper]{article}
\usepackage{authblk,xspace, amsmath, amssymb, subfigure, a4wide}
\usepackage{stmaryrd,algorithm}
\usepackage[noend]{algpseudocode}
\def\enso{Ens\={o}\xspace}
\begin{document}

\title{Information Grammars} % Schema Grammars
\author[1]{William R. Cook} 
\author[2]{Tijs van der Storm}
\affil[1]{University of Texas at Austin} 
\affil[2]{CWI}
\maketitle

\begin{abstract}
  \enso is a programming environment based on interpretation of
  domain-specific languages (DSLs). The abstract syntax of a DSL is
  represented as an object-oriented information model. We present a
  detailed description of \enso grammars, which are used to load such
  models from textual representation, similar to how the Lisp
  \texttt{read} function parses text into s-expressions.
\end{abstract}

\section{Introduction}


\begin{itemize}
\item Interpretive
\item Models not ASTs
\item General
\item Self-describing
\end{itemize}

Interpretive: parsing is interpretive, (we could do interpreter
composition to extend the interpreter for regular operators, for
instance; this could also simplify the bootstrap grammar if we can
rewrite it to not use iter\_star etc.)

Example: schema + grammar.

\section{Meta Syntax}

\def\literal#1{\ensuremath{\text{\texttt{"}}#1\text{\texttt{"}}}}

\begin{figure}
\begin{minipage}[t]{0.5\textwidth}
\[
\begin{array}{rcll}
S & ::= & \varepsilon & \text{(Empty)} \\
  & \mid & t & \text{(Token; } t\in\{\text{\texttt{str}}, \text{\texttt{int}}, \text{\texttt{sym}}, \literal{s},... \}\text{)} \\
  & \mid & N & \text{(Non-terminal)}\\
  & \mid & P\Uparrow & \text{(Reference)}\\
  & \mid & S\;\oplus & \text{(Regular; }\oplus\in\{?,*,+\}\text{)}\\ 
  & \mid & f:S & \text{(Field)}\\
  & \mid & S\cdot S & \text{(Sequence)}\\
  & \mid & [C]\; S & \text{(Create)}\\
  & \mid & S \mid S & \text{(Alternative)}\\
\end{array}
\]
\end{minipage}
\begin{minipage}[t]{0.5\textwidth}
\[
\begin{array}{rcll}
P & ::= & \varepsilon & \text{(Root)}\\
  & \mid & . & \text{(Current)}\\
%  & \mid & .. & \text{(Parent)}\\
  & \mid & P / f & \text{(Field)}\\
  & \mid & P [c] & \text{(Constant subscript)}\\
  & \mid & P [P] & \text{(Path subscript)}\\
\end{array}
\]
\end{minipage}
\caption{Abstract syntax of \enso grammars\label{FIG:metasyntax}}
\end{figure}

\begin{figure}
\begin{minipage}{0.5\textwidth}
\[
\begin{array}{rcll}
T & ::=  &\varepsilon & \text{(Empty)}\\
  & \mid & t: s & \text{(Token)}\\
  & \mid & \text{appl}(S, \overline{T}) & \text{(Application)}\\
\end{array}
\]
\end{minipage}
\begin{minipage}{0.5\textwidth}
\[
\begin{array}{rcll}
V & ::=  & \bot & \text{(Nil)}\\
  & \mid & s  & \text{(String)}\\
  & \mid & P  & \text{(Reference)}\\
  & \mid & \llbracket C \rrbracket(\overline{B}) & \text{(Object)}\\
B & ::= & f: \overline{V} & \text{(Binding)}\\
\end{array}
\]
\end{minipage}
\caption{Syntax of derivation trees $T$ and instance trees $V$\label{FIG:derivationInstancs}}
\end{figure}









\def\implode{\ensuremath{\mathcal{I}}}

\begin{figure}
\begin{minipage}[t]{0.4\textwidth}
\begin{align*}
\implode(\varepsilon, v) &= \bot\\
\implode(t: s, v) &= s\\
\implode(\text{appl}(P\Uparrow, t: s), v) &= P[\text{\texttt{it}}\mapsto s]\\
\implode(\text{appl}([C] S, \overline{T}), v) &=
\text{fold}(\implode, \overline{T}, \llbracket C\rrbracket())\\
\end{align*}
\end{minipage}
\begin{minipage}[t]{0.6\textwidth}
\begin{align*}
\implode(\text{appl}(f:S, \overline{T}), 
\llbracket C\rrbracket(\overline{B})) 
&= \llbracket C\rrbracket(\overline{B} \cup f: \sum_{\tau\in\overline{T}}\implode(\tau))\\
\implode(\text{appl}(S,\overline{T}), v) &= 
\sum_{\tau\in \overline{T}}\implode(\tau)\text{ otherwise }
\end{align*}
\end{minipage}
\caption{Implode $\implode: T\times V\rightarrow V*$: Map derivation
  trees to instance trees.\label{FIG:implode}}
\end{figure}

NB: the union in (5) is union on maps, which concatenates values for the
same key, if any. Summation is flatmap.


\algblockdefx[Case]{Case}{EndCase}[1]{\textbf{case}\  #1\ \textbf{of}}[0]{\textbf{end case}}
\algcblockdefx[Case]{Case}{When}{EndCase}[1]{\textbf{when}\  #1:}[0]{\textbf{end case}}
\algcblockdefx[Case]{Case}{Else}{EndCase}[0]{\textbf{else}}[0]{\textbf{end case}}
%\algcblock{Switch}{Case}{Esac}
Build

\begin{algorithm}
\begin{center}
%\begin{minipage}{0.7\textwidth}
\begin{algorithmic}[1]
\Function{instantiate}{$v: V$, $S: Schema$}
  \Procedure{update}{$o: S$, $f_S: Schema.classes[Field]$, $x: S$}
          \If{$f_S.many$}
            $o'[f_S.name] \ll x$ 
               \Comment{Insert instead of assign if $f$ is many-valued}
          \Else
           \ $o'[f_S.name] \gets x$
          \EndIf     
  \EndProcedure
  \State $fixups \gets \{\}$
  \Function{build}{$v: V$, $o: S$, $f_S: Schema.classes[Field]$}
    \Case{$v$}
    \When{$\llbracket C\rrbracket(\overline{B})$} 
        $o'\gets \text{\textbf{new }}S.classes[C]$
          \Comment{Create a new object of class $C$}
       \ForAll{$f:\overline{V}\in \overline{B}$, $v\in \overline{V}$}
          \State $f_S \gets S.classes[C].fields[f]$
             \Comment{Lookup $f$ in the schema $S$}
          \State \Call{update}{$o'$, $f_S$, \textsc{build}($v$, $o'$, $f_S$)}
       \EndFor
       \State \Return{$o'$}
    \When{$P$}
        $fixups \ll \langle o, f_S, P\rangle$
         \Comment{Record a fixup for later patching}
       \State \Return{$\bot$}
    \Else
       \ \Return{}\Call{valueOf}{$v$, $f_S.type$}
         \Comment{Convert the value according to the field type}
    \EndCase
  \EndFunction
  \State $o \gets$ \Call{build}{$v$, $\bot$, $\bot$}
  \ForAll{$\langle o', f_S, P\rangle\in \Call{tsort}{fixups}$}
    \State \Call{update}{$o'$, $f_S$, \textsc{deref}($o$, $o'$, $P$)}
  \EndFor
  \State \Return{$o$}
\EndFunction
\end{algorithmic}
%\end{minipage}
\end{center}
\caption{Instantiation algorithm\label{ALG:instantiate}}
\end{algorithm}

\section{Type checking}

\[
\mathcal{L}(G) = \{ \tau\in T \mid start(G) \Rightarrow^* \tau \}
\]

The goal of type checking is something like

\[
typecheck(G, S) \implies \forall \tau \in \mathcal{L}(G): 
instantiate(\implode(\tau, \bot), S) \in S
\]
[Need to be expressed in terms of derivations, an non-ambiguity]


\def\myjoin{\ensuremath{\mid}}
\def\mymeet{\ensuremath{\cdot}}

\begin{table}[ht]
\footnotesize
\centering
\subfigure[Definition of \myjoin]{
% table for join (+)
\begin{tabular}{|c||c|c|c|c|c|}\hline
$\mid$ & $0$ & $1$ & $?$ & $*$ & $+$ \\\hline\hline
$0$ & $0$ & $?$ & $?$ & $*$ & $*$ \\\hline
$1$ & $?$ & $1$ & $?$ & $*$ & $+$ \\\hline
$?$ & $?$ & $?$ & $?$ & $*$ & $*$ \\\hline
$*$ & $*$ & $*$ & $*$ & $*$ & $*$ \\\hline
$+$ & $*$ & $+$ & $*$ & $*$ & $+$ \\\hline
\end{tabular}
\label{TBL:multJoin}
}
\subfigure[Definition of \mymeet]{
% table for meet (*)
\begin{tabular}{|c||c|c|c|c|c|}\hline
$\cdot$ & $0$ & $1$ & $?$ & $*$ & $+$ \\\hline\hline
$0$ & $0$ & $1$ & $?$ & $*$ & $+$ \\\hline
$1$ & $1$ & $+$ & $+$ & $+$ & $+$ \\\hline
$?$ & $?$ & $+$ & $*$ & $*$ & $+$ \\\hline
$*$ & $*$ & $+$ & $*$ & $*$ & $+$ \\\hline
$+$ & $+$ & $+$ & $+$ & $+$ & $+$ \\\hline
\end{tabular}
\label{TBL:multMeet}
}
\subfigure[Regular operators]{
\def\mystar{*}
\begin{tabular}{|c||c|c|c|}\hline
$x$ & $x^?$ & $x^*$ & $x^+$\\\hline\hline
$0$ & 0 & 0 & 0 \\\hline  
$1$ & ? & * & + \\  \hline
$?$ & ? & * & * \\  \hline
$*$ & * & * & * \\  \hline
$+$ & * & * & + \\  \hline
\end{tabular}
\label{TBL:multRegular}
}

\caption{Algebra of multiplicities\label{TBL:multAlgebra}}
\end{table}





\def\fixpoint{\ensuremath{\text{lfp}}}
\def\reachability{\ensuremath{\mathcal{R}}}
\begin{figure}
\begin{minipage}[t]{0.5\textwidth}
\begin{align*}
\reachability_C(A \circ B) &= \reachability_C(A) \cup \reachability_C(B)
 \text{ where } \circ \in \{\mid, \cdot\}\\
\reachability_C(A\;\oplus) &= \reachability(A) \text{ where } \oplus \in \{?, *, +\}\\
\reachability_C(N) &= \fixpoint(\emptyset, \reachability_C, \alpha) \text{ where }N ::= \alpha\\
\end{align*}
\end{minipage}
\begin{minipage}[t]{0.5\textwidth}
\begin{align*}
\reachability_C(f: A) &= \{\langle C, f\rangle\} \cup \reachability_C(A)\\
\reachability_C([C'] A) &= \reachability_{C'}(A)\\
\reachability_C(x) &= \emptyset \text{ otherwise }
\end{align*}
\end{minipage}
\caption{The reachability function $\reachability_C$ computes the set
  of fields that can be derived from a grammar symbol.\label{FIG:reachability}}
\end{figure}


\def\mymultjoin{\ensuremath{\myjoin}}
\def\mymultmeet{\ensuremath{\mymeet}}
\def\mymultbottom{\ensuremath{1}}
\def\multof{\ensuremath{\mathcal{M}}}
\def\multoff{\ensuremath{\multof_f}}

\begin{figure}
\begin{minipage}[t]{0.5\textwidth}
\begin{align*}
\multoff(A\mid B) &= \multoff(A) \mymultjoin  \multoff(B)\\
\multoff(A\cdot B) &= \multoff(A) \mymultmeet  \multoff(B)\\
\multoff(A\;\oplus) &= \multoff(A)^\oplus \text{ where } \oplus\in \{?, *, +\}\\
\multoff(N) &= \fixpoint(\mymultbottom, \multoff, \alpha) \text{ where } N ::= \alpha\\
\multoff(f: A) &= \multoff(A)\\
\end{align*}
\end{minipage}
\begin{minipage}[t]{0.5\textwidth}
\begin{align*}
\multoff([C'] A) &= 1\\
\multoff(P\Uparrow) &= 1\\
\multoff(t) &= 1 \text{ where } t\in \{\text{\texttt{str}}, \text{\texttt{int}}, \text{\texttt{sym}}, \literal{s}, ...  \}\\
\multoff(x) &= 0 \text{ otherwise}
\end{align*}
\end{minipage}
\caption{The multiplicity function \multoff\ computes the multiplicity
  of field occurrences in a grammar symbol.\label{FIG:multiplicity}}
\end{figure}


\def\mytypejoin{\ensuremath{\sqcup}}
\def\mytypemeet{\ensuremath{\sqcap}}
\def\mytypebottom{\ensuremath{\bot}}
\def\mytypetop{\ensuremath{\top}}
\def\typeof{\ensuremath{\mathcal{T}}}
\def\typeoff{\ensuremath{\typeof^C_f}}
\begin{figure}
\begin{minipage}[t]{0.4\textwidth}
\begin{align*}
\typeoff(A\mid B) &= \typeoff(A) \mytypejoin  \typeoff(B)\\
\typeoff(A\cdot B) &= \typeoff(A) \mytypemeet  \typeoff(B)\\
\typeoff(A\;\oplus) &= \typeoff(A) \text{ where } \oplus \in \{?, *, +\}\\
\typeoff(N) &= \fixpoint(\mytypebottom, \typeoff, \alpha) \text{ where } N ::= \alpha\\
\typeoff(f: A) &= \typeoff(A)\\
\typeoff([C'] A) &= \sigma.classes[C']\\
\end{align*}
\end{minipage}
\begin{minipage}[t]{0.6\textwidth}
\begin{align*}
\typeoff(\Uparrow) &= root\\
\typeoff(/P\Uparrow) &= \text{deref}(root, P)\\
\typeoff(./P\Uparrow) &= \text{deref}(C, P)\\
\typeoff(t) &= \sigma.prims[\text{\texttt{str}}] \text{ where } t\in\{\text{\texttt{sym}}, \literal{s}, ...\}\\
\typeoff(t) &= \sigma.prims[t] \text{ where } t\in \{\text{\texttt{str}}, \text{\texttt{int}}, \text{\texttt{real}} \}\\
\typeoff(x) &= \mytypetop \text{ otherwise}
\end{align*}
\end{minipage}
\caption{Function \typeoff computes the schema type of the field named
  $f$ in class $C$ as used in a grammar.\label{FIG:typeof}}
\end{figure}


% F, such that \bigsqcup_{i\in \mathbb{N}} F^i(S \mapsto \emptyset)

\section{Parsing}

General: left-recursion etc., infinite lookahead. GLL. 

Fixed tokenizer: builtin paths for references, identifiers, int, real,
string, whitespace and comments. Escaping of identifiers to avoid
clashes. Automatic reservation of (alpha-numeric) literals. Code
symbols (@"code"). NB: naive implementation; could go scannerless, but
need to implement lookahead restrictions on identifiers and whitespace
to avoid ambiguities.

\begin{verbatim}
class EvalGrammar < Eval
  def Item(this, gll, _)
    return gll.pop if this.finished?
    eval(this.nxt, gll, this.move)
  end

  def Alt(this, gll, nxt)
    item = Item.new(this)
    if this.empty? then
      gll.empty_node(item)
      eval(nxt, gll, nil) if nxt
    else
      gll.create(nxt) if nxt
      eval(item, gll, nil)
    end
  end
  
  def Rule(this, gll, nxt)
    gll.create(nxt) if nxt
    this.alts.each do |alt|
      gll.add(alt)
    end
  end

  def Token(this, gll, nxt)
    return unless this.match?(gll.token)
    cr = gll.leaf_node(this)
    if nxt then
      gll.item_node(nxt, cr)
      eval(nxt, gll, nil)
    end
  end
end
\end{verbatim}


\section{Instantiating}

References: xText calls this their unique feature, and it is almost
completely the same as in Enso, even the fact that refs refer to
schema classes, not non-terminals.... However, it might provide an
interesting point for the bootstrapping, since grammar.grammar also
uses it.

No AST, but schema instance. Asynchronous transformation/binding. Case
in point (?): all binary expressions map to BinExp class, op:"+" sets
the operator literal. Other case in point: populating a collection
field, using your own recursive rule (TODO: check that it works this
way, i.e. a:X* does it append the Xs to a or set them? It should be
monotone, never overwrite anything, for collections this should mean
append). Way it works: intermediate, generic tree (also a model; the
instance tree); one visit to construct objects and populate fields,
final fixup phase for cross refs. NB: xText does it the same, but no
details, and Node tree is not a model.

\section{Modularity}

Modular: merge on both schema and grammar. Should we fully discuss
merge here? Can we modularly type check 2 grammars against 2 schemas
each, and ensure that the merged composition will type check? Abstract
rules. Modularity enabled by general parsing strategy + fixed
tokenizer (however, more keywords will be reserved).

Abstract rules. Merge. Escaping of keywords.


\section{Bootstrapping}

Bootstrapped/self-describing: initial grammar in
ruby. Grammar.grammar, grammar.schema etc. This is a major
(implementation) different with xText.

\section{Origin Tracking}

Model origin tracking: distinguish ref and reffed origin (Maybe too
much engineering issue, but possibly interesting because of merging
Location model into Instance model etc.

\section{Evaluation}

List languages used. Performance??? LOC of framework.


\section{Conclusion}






\end{document}
% Lisp reader n. Trad. the procedure that parses character representations of objects from a stream, producing objects. (This procedure is implemented by the function read.)
%http://www.lispworks.com/documentation/HyperSpec/Body/26_glo_l.htm#lisp_reader


Typechecking: TODO. I think this would be an interesting extra
interpreter.

Debugging: could we do a simple grammar debugger by interpreter
composition? Our interpreter approach really makes debugging output
easy.

Schema inference: TODO (Do we want this? If this is really easy to
implement, we can use it to compare with xText.) Other way around:
derive a grammar from a schema.



Bi-directional: rendering/layouting. Key thing: constructor directives
become predicates, and assignments become predicates (also in @"" code
things; sequencing is conjunction). Interesting point: insertion of
required parens, with expressions.
